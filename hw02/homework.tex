\documentclass[portrait,a4paper]{article}


\usepackage[utf8x]{inputenc}
\usepackage[T1]{fontenc}


\usepackage{mathtools}
\usepackage{amssymb,amsfonts,amsmath}
\usepackage[e]{esvect}

\usepackage{algorithmic}
\usepackage{algorithm}
\newcommand{\algorithmlabel}[2]{{
    \renewcommand{\algorithmicensure}{\textbf{#1}:}
    \ENSURE{#2~}
}}

\usepackage{graphicx}
\usepackage[svgnames]{xcolor}

\usepackage{geometry}
\geometry{a4paper}

% multirow and multicol
\usepackage{multirow}
\usepackage{multicol}
\columnsep24pt
\columnseprule0.1pt

% enumerate
\renewcommand\theenumi{\arabic{enumi}}
\renewcommand\labelenumi{\theenumi.}
\renewcommand\theenumii{\roman{enumii}}
\renewcommand\labelenumii{\theenumii)}

\usepackage{listings}
\lstset{
    floatplacement={tbp},
    basicstyle=\ttfamily\mdseries,
    identifierstyle=,
    stringstyle=\color{gray},
    numbers=left,
    numbersep=5pt,
    inputencoding=utf8x,
    xleftmargin=8pt,
    xrightmargin=8pt,
    keywordstyle=[1]\bfseries,
    keywordstyle=[2]\bfseries,
    keywordstyle=[3]\bfseries,
    keywordstyle=[4]\bfseries,
    numberstyle=\tiny,
    stepnumber=1,
    breaklines=true,
    frame=lines,
    showstringspaces=false,
    tabsize=2,
    commentstyle=\color{gray},
    captionpos=b,
    float=float,
    language={Java}
}
\newcommand{\code}[1]{\lstinline{#1}}

% hyperref
\usepackage[colorlinks=true,pdfborder={0 0 0},citecolor=DarkGreen,linkcolor=DarkBlue,urlcolor=DarkBlue]{hyperref}

% depth of section numbering
\setcounter{secnumdepth}{4}

% redefine the \paragraph command:
\makeatletter
\renewcommand\paragraph{\@startsection{paragraph}{4}{0mm}%
    {-\baselineskip}%
    {0.5\baselineskip}%
    {\normalfont\bfseries}%
}%
\makeatother

% new chapter command
\newcommand{\newchapter}{\clearpage\pagebreak}

% theorems
\usepackage{amsthm}
\newtheorem{lemma}{Lemma}[section]
\newtheorem{theorem}{Theorem}[section]
\newtheorem{corollary}{Corollary}[section]
\newtheorem{definition}{Definition}[section]
\newtheorem{remark}{Remark}[section]
\newtheorem{observation}{Observation}[section]
\newtheorem{assumption}{Assumption}[section]
\newtheorem{proposition}{Proposition}[section]

% autoref names
\newcommand{\specialref}[2]{\hyperref[#1]{#2~\ref*{#1}}}
\def\lstlistingautorefname{Listing}
\def\subsubsectionautorefname{Section}
\def\subsectionautorefname{Section}
\def\figureautorefname{Figure}

% parindent
\parindent0px
\parskip3pt

% redefine greek letters
\renewcommand{\phi}{\varphi}
\renewcommand{\epsilon}{\varepsilon}

% shortcuts in math mode
% \newcommand{\bs}{\boldsymbol}
\newcommand{\mc}{\mathcal}
\newcommand{\ds}{\displaystyle}
\DeclarePairedDelimiter\absimpl{\lvert}{\rvert}
\DeclarePairedDelimiter\normimpl{\lVert}{\rVert}
\newcommand{\abs}[1]{\absimpl*{#1}}
\newcommand{\norm}[1]{\normimpl*{#1}}
\newcommand{\argmax}{\operatorname*{arg\,max}}
\newcommand{\argmin}{\operatorname*{arg\,min}}

% number sets
\newcommand{\R}{\mathbb{R}}
\newcommand{\Z}{\mathbb{Z}}
\newcommand{\N}{\mathbb{N}}
\newcommand{\Q}{\mathbb{Q}}
\newcommand{\C}{\mathbb{C}}
\newcommand{\F}{\mathbb{F}}
\newcommand{\LL}{\mathcal{L}}
\newcommand{\powerset}{\mathcal P}

% probabilities
\newcommand{\Prob}[1]{\operatorname{Pr}\left[#1\right]}
\newcommand{\Ex}[1]{\mathbb{E}\left[#1\right]}

% misc
\newcommand{\bigO}[1]{\mc O\left(#1\right)} % big-o notation

\newcommand{\nop}[1]{} % temporarily remove from output




% todo
\usepackage{framed}
\newenvironment{todo}
{\color{DarkRed} \begin{leftbar}}
{\end{leftbar}}
\newcommand{\inlinetodo}[1]{{\textcolor{DarkRed}{ [\textbf{TODO}: #1]}}}
\newcommand{\mat}[1]{\bs{#1}}
\newcommand{\ma}{\mat{A}}
\newcommand{\mb}{\mat{B}}
\newcommand{\mx}{\mat{X}}
\newcommand{\mv}{\mat{V}}
\newcommand{\muu}{\mat{U}}
\newcommand{\md}{\mat{D}}
\newcommand{\ms}{\mat{S}}
\newcommand{\mz}{\mat{Z}}

\newcommand{\vx}{\mat{x}}
\newcommand{\va}{\mat{a}}
\newcommand{\vb}{\mat{b}}
\newcommand{\vu}{\mat{u}}
\newcommand{\vz}{\mat{z}}

\newcommand{\rd}{\R^D}
\newcommand{\rr}[2]{\R^{#1 \times #2}}

\usepackage{placeins}

\newcommand*{\titleSW}[3]{\begingroup% Story of Writing
\raggedleft
\vspace*{\baselineskip}
{\Huge\textbf{#1}}\\[0.7\baselineskip]
{\large\textbf{#2}}\\[0.5\baselineskip]
{\small #3}\par
\endgroup}

\begin{document}

 \author{Pascal Spörri\\pascal@spoerri.io}
 \title{HOWTO WRITE FAST NUMERICAL CODE\\ EXERCISE 2}
 \date{\today}
\maketitle

\section{Project Information}
\section{Microbenchmarks}
For this exercise we had to benchmark several mathematical functions:
\begin{itemize}
    \item $y=\sin(x)$, C Function: \lstinline{y=sin(x)}
    \item $y=\log(x+0.1)$, C Function: \lstinline{y=log(x+0.1)}
    \item $y=e^x$, C Function: \lstinline{y=exp(x)}
    \item $y={1\over x+1}$, C Function: \lstinline{y=1.0/(x+1.0)}
    \item $y=x^2$, C Function: \lstinline{y=x*x}
\end{itemize}
Since we benchmark on OSX we had the problem that we couldn't use \lstinline{-march=corei7-avx} since the provided Apple assembler is unable to generate AVX code. Using a tip\footnote{\url{http://old.nabble.com/Re\%3a-gcc,-as,-AVX,-binutils-and-MacOS-X-10.7-p32584737.html}} we were able to replace the \lstinline{as} program on our machines with the assembler from clang. 

\paragraph*{System Setup:}
\begin{description}
    \item[Compiler:] gcc-4.7 (GCC) 4.7.2
    \item[Assembler:] Apple clang version 4.1 (tags/Apple/clang-421.11.66) (based on LLVM 3.1svn)
    \item[Operating System:] Mac OSX 10.8.2
    \item[CPU:] Intel(R) Core(TM) i7-3720QM CPU @ 2.60GHz
\end{description}

We benchmarked our code with the following flags enabled: \lstinline{-O3 -m64 -march=corei7-avx -fno-tree-vectorize}. We deliberately disabled vectorization since the automatic vectorization support for GCC is perceived as poor and we wanted to get explainable results.
%Running tests with 0
%sin 0 8.89287
%log 0 22.2584
%exp 0 11.1266
%oneover 0 6.25691
%squared 0 1.61234
%Running tests with 0.9
%sin 0.9 32.2468
%log 0.9 20.8129
%exp 0.9 20.6802
%oneover 0.9 10.3679
%squared 0.9 1.50901
%Running tests with 1.1
%sin 1.1 32.593
%log 1.1 20.948
%exp 1.1 23.1887
%oneover 1.1 10.4601
%squared 1.1 1.61984
%Running tests with 4.12345
%sin 4.12345 30.6983
%log 4.12345 25.9272
%exp 4.12345 23.4752
%oneover 4.12345 10.6315
%squared 4.12345 1.63883
\begin{figure}[H]
    \centering
    \begin{tabular}{l||r|r|r|r|r}
        \textbf{Function} & $x=0$ & $x=0.9$ & $x=1.1$ & $x=4.12345$ \\ \hline
        $y=\sin(x)$     & $8.89$ & $32.25$& $32.59$& $30.70$ \\
        $y=\log(x+0.1)$ & $22.26$ & $20.81$ & $20.95$ & $25.93$\\
        $y=e^x$         & $11.13$ & $20.68$ & $23.19$ & $23.48$\\
        $y={1\over x+1}$& $6.26$ & $10.36$ & $10.46$ & $10.63$\\
        $y=x^2$         & $1.61$  & $1.50$  & $1.62$  & $1.64$
    \end{tabular}
    \caption{Timings in cycles per mathematical function using \lstinline{-O3 -m64 -march=corei7-avx  -fno-tree-vectorize} with GCC 4.7.2.}
\end{figure}

\subsection{Observations}
\begin{itemize}
    \item $y=\sin(x)$: We observe that we require significantly less cycles for $sin(0)$ than for the different function values of $\neq 0$. The library can make use of the approximation $\sin(\theta)\approx \theta$ for a significantly small theta. 
    \item $y=\log(x+0.1)$: We don't observe a significant change between the different function values. 
    \item $y=e^x$: The CPU is able to make use of a direct computation for $x=0$. 
    \item $y={1\over x+1}$: The CPU is able to identify the special condition ${1\over 1}$.
    \item $y=x^2$: No change over the different inputs. The cycle count indicates pipelining.
\end{itemize}









\subsection{GCC}
\end{document}
