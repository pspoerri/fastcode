\documentclass[portrait,a4paper]{article}


\usepackage[utf8x]{inputenc}
\usepackage[T1]{fontenc}


\usepackage{mathtools}
\usepackage{amssymb,amsfonts,amsmath}
\usepackage[e]{esvect}

\usepackage{algorithmic}
\usepackage{algorithm}
\newcommand{\algorithmlabel}[2]{{
    \renewcommand{\algorithmicensure}{\textbf{#1}:}
    \ENSURE{#2~}
}}

\usepackage{graphicx}
\usepackage[svgnames]{xcolor}

\usepackage{geometry}
\geometry{a4paper}

% multirow and multicol
\usepackage{multirow}
\usepackage{multicol}
\columnsep24pt
\columnseprule0.1pt

% enumerate
\renewcommand\theenumi{\arabic{enumi}}
\renewcommand\labelenumi{\theenumi.}
\renewcommand\theenumii{\roman{enumii}}
\renewcommand\labelenumii{\theenumii)}

\usepackage{listings}
\lstset{
    floatplacement={tbp},
    basicstyle=\ttfamily\mdseries,
    identifierstyle=,
    stringstyle=\color{gray},
    numbers=left,
    numbersep=5pt,
    inputencoding=utf8x,
    xleftmargin=8pt,
    xrightmargin=8pt,
    keywordstyle=[1]\bfseries,
    keywordstyle=[2]\bfseries,
    keywordstyle=[3]\bfseries,
    keywordstyle=[4]\bfseries,
    numberstyle=\tiny,
    stepnumber=1,
    breaklines=true,
    frame=lines,
    showstringspaces=false,
    tabsize=2,
    commentstyle=\color{gray},
    captionpos=b,
    float=float,
    language={Java}
}
\newcommand{\code}[1]{\lstinline{#1}}

% hyperref
\usepackage[colorlinks=true,pdfborder={0 0 0},citecolor=DarkGreen,linkcolor=DarkBlue,urlcolor=DarkBlue]{hyperref}

% depth of section numbering
\setcounter{secnumdepth}{4}

% redefine the \paragraph command:
\makeatletter
\renewcommand\paragraph{\@startsection{paragraph}{4}{0mm}%
    {-\baselineskip}%
    {0.5\baselineskip}%
    {\normalfont\bfseries}%
}%
\makeatother

% new chapter command
\newcommand{\newchapter}{\clearpage\pagebreak}

% theorems
\usepackage{amsthm}
\newtheorem{lemma}{Lemma}[section]
\newtheorem{theorem}{Theorem}[section]
\newtheorem{corollary}{Corollary}[section]
\newtheorem{definition}{Definition}[section]
\newtheorem{remark}{Remark}[section]
\newtheorem{observation}{Observation}[section]
\newtheorem{assumption}{Assumption}[section]
\newtheorem{proposition}{Proposition}[section]

% autoref names
\newcommand{\specialref}[2]{\hyperref[#1]{#2~\ref*{#1}}}
\def\lstlistingautorefname{Listing}
\def\subsubsectionautorefname{Section}
\def\subsectionautorefname{Section}
\def\figureautorefname{Figure}

% parindent
\parindent0px
\parskip3pt

% redefine greek letters
\renewcommand{\phi}{\varphi}
\renewcommand{\epsilon}{\varepsilon}

% shortcuts in math mode
% \newcommand{\bs}{\boldsymbol}
\newcommand{\mc}{\mathcal}
\newcommand{\ds}{\displaystyle}
\DeclarePairedDelimiter\absimpl{\lvert}{\rvert}
\DeclarePairedDelimiter\normimpl{\lVert}{\rVert}
\newcommand{\abs}[1]{\absimpl*{#1}}
\newcommand{\norm}[1]{\normimpl*{#1}}
\newcommand{\argmax}{\operatorname*{arg\,max}}
\newcommand{\argmin}{\operatorname*{arg\,min}}

% number sets
\newcommand{\R}{\mathbb{R}}
\newcommand{\Z}{\mathbb{Z}}
\newcommand{\N}{\mathbb{N}}
\newcommand{\Q}{\mathbb{Q}}
\newcommand{\C}{\mathbb{C}}
\newcommand{\F}{\mathbb{F}}
\newcommand{\LL}{\mathcal{L}}
\newcommand{\powerset}{\mathcal P}

% probabilities
\newcommand{\Prob}[1]{\operatorname{Pr}\left[#1\right]}
\newcommand{\Ex}[1]{\mathbb{E}\left[#1\right]}

% misc
\newcommand{\bigO}[1]{\mc O\left(#1\right)} % big-o notation

\newcommand{\nop}[1]{} % temporarily remove from output




% todo
\usepackage{framed}
\newenvironment{todo}
{\color{DarkRed} \begin{leftbar}}
{\end{leftbar}}
\newcommand{\inlinetodo}[1]{{\textcolor{DarkRed}{ [\textbf{TODO}: #1]}}}
\newcommand{\mat}[1]{\bs{#1}}
\newcommand{\ma}{\mat{A}}
\newcommand{\mb}{\mat{B}}
\newcommand{\mx}{\mat{X}}
\newcommand{\mv}{\mat{V}}
\newcommand{\muu}{\mat{U}}
\newcommand{\md}{\mat{D}}
\newcommand{\ms}{\mat{S}}
\newcommand{\mz}{\mat{Z}}

\newcommand{\vx}{\mat{x}}
\newcommand{\va}{\mat{a}}
\newcommand{\vb}{\mat{b}}
\newcommand{\vu}{\mat{u}}
\newcommand{\vz}{\mat{z}}

\newcommand{\rd}{\R^D}
\newcommand{\rr}[2]{\R^{#1 \times #2}}

\usepackage{placeins}

\newcommand*{\titleSW}[3]{\begingroup% Story of Writing
\raggedleft
\vspace*{\baselineskip}
{\Huge\textbf{#1}}\\[0.7\baselineskip]
{\large\textbf{#2}}\\[0.5\baselineskip]
{\small #3}\par
\endgroup}

\begin{document}

 \author{Pascal Spörri\\pascal@spoerri.io}
 \title{HOWTO WRITE FAST NUMERICAL CODE\\ EXERCISE 2}
 \date{\today}
\maketitle

\section{Project Information}
Adrian Blumer will provide the project information.
\section{Microbenchmarks}
For this exercise we had to benchmark several mathematical functions:
\begin{itemize}
    \item $y=\sin(x)$, C Function: \lstinline{y=sin(x)}
    \item $y=\log(x+0.1)$, C Function: \lstinline{y=log(x+0.1)}
    \item $y=e^x$, C Function: \lstinline{y=exp(x)}
    \item $y={1\over x+1}$, C Function: \lstinline{y=1.0/(x+1.0)}
    \item $y=x^2$, C Function: \lstinline{y=x*x}
\end{itemize}
Since we benchmark on OSX we had the problem that we couldn't use \lstinline{-march=corei7-avx} since the provided Apple assembler is unable to generate AVX code. Using a tip\footnote{\url{http://old.nabble.com/Re\%3a-gcc,-as,-AVX,-binutils-and-MacOS-X-10.7-p32584737.html}} we were able to replace the \lstinline{as} program on our machines with the assembler from clang. 

\paragraph*{System Setup:}
\begin{description}
    \item[Compiler:] gcc-4.7 (GCC) 4.7.2
    \item[Assembler:] Apple clang version 4.1 (tags/Apple/clang-421.11.66) (based on LLVM 3.1svn)
    \item[Operating System:] Mac OSX 10.8.2
    \item[CPU:] Intel(R) Core(TM) i7-3720QM CPU @ 2.60GHz
\end{description}

We benchmarked our code with the following flags enabled: \lstinline{-O3 -m64 -march=corei7-avx -fno-tree-vectorize}. We deliberately disabled vectorization since the automatic vectorization support for GCC is perceived as poor and we wanted to get explainable results.
%Running tests with 0
%sin 0 8.89287
%log 0 22.2584
%exp 0 11.1266
%oneover 0 6.25691
%squared 0 1.61234
%Running tests with 0.9
%sin 0.9 32.2468
%log 0.9 20.8129
%exp 0.9 20.6802
%oneover 0.9 10.3679
%squared 0.9 1.50901
%Running tests with 1.1
%sin 1.1 32.593
%log 1.1 20.948
%exp 1.1 23.1887
%oneover 1.1 10.4601
%squared 1.1 1.61984
%Running tests with 4.12345
%sin 4.12345 30.6983
%log 4.12345 25.9272
%exp 4.12345 23.4752
%oneover 4.12345 10.6315
%squared 4.12345 1.63883
\begin{figure}[H]
    \centering
    \begin{tabular}{l|r|r|r|r}
        \textbf{Function} & $x=0$ & $x=0.9$ & $x=1.1$ & $x=4.12345$ \\ \hline
        $y=\sin(x)$     & $8.89$ & $32.25$& $32.59$& $30.70$ \\
        $y=\log(x+0.1)$ & $22.26$ & $20.81$ & $20.95$ & $25.93$\\
        $y=e^x$         & $11.13$ & $20.68$ & $23.19$ & $23.48$\\
        $y={1\over x+1}$& $6.26$ & $10.36$ & $10.46$ & $10.63$\\
        $y=x^2$         & $1.61$  & $1.50$  & $1.62$  & $1.64$
    \end{tabular}
    \caption{Timings in cycles per mathematical function using \lstinline{-O3 -m64 -march=corei7-avx  -fno-tree-vectorize} with GCC 4.7.2.}
\end{figure}

\subsection{Observations}
\begin{itemize}
    \item $y=\sin(x)$: We observe that we require significantly less cycles for $sin(0)$ than for the different function values of $\neq 0$. The library can make use of the approximation $\sin(\theta)\approx \theta$ for a significantly small theta. 
    \item $y=\log(x+0.1)$: We don't observe a significant change between the different function values. 
    \item $y=e^x$: The CPU is able to make use of a direct computation for $x=0$. 
    \item $y={1\over x+1}$: The CPU is able to identify the special condition ${1\over 1}$.
    \item $y=x^2$: No change over the different inputs. The cycle count indicates pipelining.
\end{itemize}

\section{Optimization Blockers}
\subsection{Code Variants}
\subsubsection{Original Code}
The original code performed quite poorly. This is mainly due to multiple function calls and pointer references.
\subsubsection{Localized Loop Variables}
As a first step we have localized the loop variable n to avoid a memory lookup for each iteration.
\subsubsection{Inlining f}
Since the \lstinline{f} function was only accessed once per loop iteration we inlined the function to get a better overview over the code. 
\subsubsection{Replacing the inner Functions}
We removed the array from the inner loop.
\subsubsection{Reordering the inner Loops}
Looking at the code of the \lstinline{get_elt} and \lstinline{set_elt} function we noticed that
the memory accesses were not coalesced. We therefore switched the loop ordering between the inner and outer loop.
\subsubsection{Reordering cos and sin Functions}
We move the computation of the \lstinline{sin} and \lstinline{cos} functions to the outer loop since they need to be computed only once per outer loop iteration.
\subsubsection{Moving x1 and x2 into the sum}
We replaced the memory accesses in the \lstinline{sum1} and \lstinline{sum2} with the already computed values \lstinline{x1} and \lstinline{x2}.
\subsubsection{Restructuring operations}
Using the fact that $\cos(x) = \cos(-x)$.
We restructured the \lstinline{sin} and \lstinline{cos} computations such that the computations use less operations. 
\subsubsection{Inlining get and set}
Since we can safely assume that our code is correct we inlined the get and set operations of the matrix. Therefore removing the bound checks from the code.
\subsubsection{Replace \texttt{x1 + cosp * x1}}
We replaced the computation of \lstinline{x1 + cosp * x1} with \lstinline{(cosp + 1) * x1} and added the \lstinline{+1} to the outer loop.
\subsubsection{Unroll inner loop}
In order to get coalesced memory accesses we unrolled the inner loop. The code is provided in listing \ref{v10_inner_loop_unrolling}.

\lstset{language=C}
\begin{lstlisting}[caption=Improved code,label=v10_inner_loop_unrolling]
void v10_inner_loop_unrolling(smat_t *a)
{
    int i, j;
    double x1,x2,y1,y2;
    double sum1, sum2;

    double *mat = a->mat;

    int n = a->n;
    int factor = 1;
    for(i = 0; i < n; i=i+2)        
    {   
        double cosp = cos(i)+1;
        double sinp = sin(factor*i);
        factor = -factor;
        
        int p = i+1;        
        for(j = 0; j < n; j=j+2)
        {        
            x1 = mat[i * n + j];
            y1 = mat[i * n + j+1]; 
            x2 = mat[p * n + j];
            y2 = mat[p * n + j+1];

            mat[i * n + j] = cosp * x1 + sinp * x2;
            mat[i * n + j+1] = cosp * y1 + sinp * y2;
            mat[p * n + j] = cosp * x2 - sinp * x1;      
            mat[p * n + j+1] = cosp * y2 - sinp * y1;  
        }    
    }
}
\end{lstlisting}

\subsection{Benchmarking}
Due to benchmarking problems I decided to use a Linux operating system for benchmarking.
We were able to improve the performance of the code from 74 MFLOPs (average) to 3.3 GFLOPs (average) getting a speedup of $44x$. 
\paragraph*{System Setup:}
\begin{description}
    \item[Compiler:] gcc (Ubuntu/Linaro 4.6.3-1ubuntu5) 4.6.3
    \item[Compiler Options:] The two different compiler options that were used to test the code.
        \begin{itemize}
            \item Optimized Compile Flags: \lstinline{-m64 -march=corei7-avx -fno-tree-vectorize -O3}
            \item Optimizations Disabled: \lstinline{-m64 -O0}     
        \end{itemize}
    \item[Operating System:] Ubuntu 12.04
    \item[CPU:] Intel(R) Core(TM) i5-3570K CPU @ 3.40GHz
\end{description}
%%%% -m64 -march=corei7-avx -fno-tree-vectorize -O3
%superslow: original function:
%Size 300: 76.918 MFLOPs
%Size 600: 71.995 MFLOPs
%Average: 74.457 MFLOPs
%
%v1_localized_variables: Localized Loop variables:
%Size 300: 74.995 MFLOPs
%Size 600: 69.494 MFLOPs
%Average: 72.245 MFLOPs
%
%v2_inline_f: Inline f:
%Size 300: 76.266 MFLOPs
%Size 600: 69.226 MFLOPs
%Average: 72.746 MFLOPs
%
%v3_replace_inner_functions: Replace Inner functions:
%Size 300: 76.266 MFLOPs
%Size 600: 69.763 MFLOPs
%Average: 73.015 MFLOPs
%

%v4_reorder_inner_loop: Reorder Inner Loops:
%Size 300: 82.564 MFLOPs
%Size 600: 82.187 MFLOPs
%Average: 82.375 MFLOPs
%
%v5_reorder_computations: Reorder COS and SIN:
%Size 300: 346.131 MFLOPs
%Size 600: 339.600 MFLOPs
%Average: 342.865 MFLOPs
%
%v6_reorder_variables: Move x1 and x2 into sum:
%Size 300: 499.965 MFLOPs
%Size 600: 499.969 MFLOPs
%Average: 499.967 MFLOPs
%
%v7_restructure_cos_sin: Restructure cos and sin and remove modulo operation:
%Size 300: 473.653 MFLOPs
%Size 600: 499.969 MFLOPs
%Average: 486.811 MFLOPs
%

%v8_inline_get_and_set: Inline get and set functions:
%Size 300: 2249.859 MFLOPs
%Size 600: 2571.245 MFLOPs
%Average: 2410.552 MFLOPs
%

%v9_remove_add_op_cos: Replace x1+cosp*x1 with x1*(cosp+1):
%Size 300: 2249.859 MFLOPs
%Size 600: 3599.640 MFLOPs
%Average: 2924.750 MFLOPs
%
%v10_inner_loop_unrolling: Unroll inner loop:
%Size 300: 2999.750 MFLOPs
%Size 600: 3599.640 MFLOPs
%Average: 3299.695 MFLOPs
%
%Best: v10_inner_loop_unrolling: Unroll inner loop
%Perf: 3299.695 MFLOPs
%%%%% -00
%superslow: original function:
%Size 300: 29.603 MFLOPs
%Size 600: 27.776 MFLOPs
%Average: 28.690 MFLOPs
%
%v1_localized_variables: Localized Loop variables:
%Size 300: 29.799 MFLOPs
%Size 600: 28.167 MFLOPs
%Average: 28.983 MFLOPs
%
%v2_inline_f: Inline f:
%Size 300: 29.506 MFLOPs
%Size 600: 27.648 MFLOPs
%Average: 28.577 MFLOPs
%
%v3_replace_inner_functions: Replace Inner functions:
%Size 300: 46.389 MFLOPs
%Size 600: 42.957 MFLOPs
%Average: 44.673 MFLOPs
%
%v4_reorder_inner_loop: Reorder Inner Loops:
%Size 300: 49.177 MFLOPs
%Size 600: 49.177 MFLOPs
%Average: 49.177 MFLOPs
%
%v5_reorder_computations: Reorder COS and SIN:
%Size 300: 119.992 MFLOPs
%Size 600: 121.614 MFLOPs
%Average: 120.803 MFLOPs
%
%v6_reorder_variables: Move x1 and x2 into sum:
%Size 300: 152.533 MFLOPs
%Size 600: 156.512 MFLOPs
%Average: 154.522 MFLOPs
%
%v7_restructure_cos_sin: Restructure cos and sin and remove modulo operation:
%Size 300: 152.533 MFLOPs
%Size 600: 156.512 MFLOPs
%Average: 154.522 MFLOPs
%
%v8_inline_get_and_set: Inline get and set functions:
%Size 300: 473.653 MFLOPs
%Size 600: 514.253 MFLOPs
%Average: 493.953 MFLOPs
%
%v9_remove_add_op_cos: Replace x1+cosp*x1 with x1*(cosp+1):
%Size 300: 692.254 MFLOPs
%Size 600: 692.261 MFLOPs
%Average: 692.258 MFLOPs
%
%v10_inner_loop_unrolling: Unroll inner loop:
%Size 300: 692.254 MFLOPs
%Size 600: 749.953 MFLOPs
%Average: 721.104 MFLOPs
%
%Best: v10_inner_loop_unrolling: Unroll inner loop
%Perf: 721.104 MFLOPs

We run the code with the two different compile flags on the setup above:
\begin{figure}[H]
    \hspace{-20mm}
    {\small
    \centering
    \begin{tabular}{r|rrr|rrr}
        \multirow{2}{*}{\textbf{Optimizations}} & \multicolumn{3}{c}{\textbf{Optimized Compile Flags}}
            & \multicolumn{3}{|c}{\textbf{Optimizations Disabled}} \\ 
                               & Size: $300$     & Size: $600$    & \textbf{Average}
                               & Size: $300$     & Size: $600$    & \textbf{Average}  \\ \hline
        Original Function & $77$ MFlops & $72$ MFlops & $\mathbf{74}$ \textbf{MFLOPs} 
           & $29$ MFlops & $28$ MFlops & $\mathbf{28}$ \textbf{MFLOPs}\\
        Localized Loop Variables & $75$ MFlops & $69$ MFlops & $\mathbf{72}$ \textbf{MFLOPs}            
           & $30$ MFlops & $28$ MFlops & $\mathbf{29}$ \textbf{MFLOPs}\\
        Inlineing f      & $76$ MFlops & $69$ MFlops & $\mathbf{73}$ \textbf{MFLOPs}
           & $29$ MFlops & $28$ MFlops & $\mathbf{29}$ \textbf{MFLOPs}\\
        Replace Inner Functions & $76$ MFlops & $69$ MFlops & $\mathbf{73}$ \textbf{MFLOPs} 
           & $46$ MFlops & $43$ MFlops & $\mathbf{45}$ \textbf{MFLOPs}\\
        Reorder Inner Loops & $83$ MFlops & $82$ MFlops & $\mathbf{82}$ \textbf{MFLOPs}
           & $49$ MFlops & $49$ MFlops & $\mathbf{49}$ \textbf{MFLOPs}\\
        Reorder cos and sin & $346$ MFlops & $340$ MFlops & $\mathbf{343}$ \textbf{MFLOPs}
           & $120$ MFlops & $122$ MFlops & $\mathbf{121}$ \textbf{MFLOPs}\\
        Move x1 and x2 into sum & $500$ MFlops & $500$ MFlops & $\mathbf{500}$ \textbf{MFLOPs}
           & $153$ MFlops & $157$ MFlops & $\mathbf{155}$ \textbf{MFLOPs}\\
        Restructure operations  & $474$ MFlops & $500$ MFlops & $\mathbf{487}$ \textbf{MFLOPs} 
           & $153$ MFlops & $155$ MFlops & $\mathbf{154}$ \textbf{MFLOPs}\\
        Inline get and set & $2250$ MFlops & $2571$ MFlops & $\mathbf{2411}$ \textbf{MFLOPs} 
           & $500$ MFlops & $500$ MFlops & $\mathbf{500}$ \textbf{MFLOPs}\\
        Replace $x1+cosp*x1$ & $2250$ MFlops & $3600$ MFlops & $\mathbf{2925}$ \textbf{MFLOPs}
           & $692$ MFlops & $667$ MFlops & $\mathbf{679}$ \textbf{MFLOPs}\\
        Unroll inner loop & $3000$ MFlops & $3600$ MFlops & $\mathbf{3300}$ \textbf{MFLOPs} 
           & $750$ MFlops & $750$ MFlops & $\mathbf{750}$ \textbf{MFLOPs}\\
    \end{tabular}
    }
    \caption{Timings chart for the optimization blockers exercise.}
\end{figure}
We observed that the function "Unroll inner loop" (listing: \ref{v10_inner_loop_unrolling}) showed the best performance for both the unoptimized and the optimized compiler flags. We also noticed increased performance between the two different input sizes: $n=300$ and $n=600$. We believe this results from a better caching behaviour from my code with large matrices.

We also observe that the first $4$ optimisations didn't increase the performance of the code. In combination with the other optimisations especially the Inlining of the \lstinline{get} \lstinline{set} operations gave the highest performance increase.

\section{Locality of Matrix Transpositions}
In this exercise we investigate the locality of a Matrix Transposition in the form provided in listing \ref{code:matrix}.
\begin{lstlisting}[caption=Matrix transpose code,label=code:matrix]
double A[M][N], B[N][M];

for (int i=0; i < M; i++) {
    for (int j=0; j < N; j++) {
        B[j][i] = A[i][j];
    }
}
\end{lstlisting}

\begin{description}
    \item[Matrix $\mathbf{A}$:] Array accesses on $\mathbf{A}$ have spatial but no temporal locality. Since every element is only read once.
    \item[Matrix $\mathbf{B}$:] Array accesses on $\mathbf{B}$ have neither spatial nor temporal locality.\end{description}

\section{Locality of a Triangular Solver}
For this exercise we consider a solver for a triangular system of equations $Lx=b$, where $L$ is a lower triangular matrix of size $N\times N$. $b$ and $x$ are vectors of size $N$. The code is provided in listing \ref{code_solver}.

\begin{lstlisting}[caption=Code for a Triangular Solver,label=code_solver]
double L[N][N], b[N];

for (int i=0; i < N-1; i++) {
    b[i] = b[i]/L[i][i];
    for (int j = i+1; j < N; j++) {
        b[j] -= b[i]*L[j][i];
    }
}
b[N-1] = b[N-1]/L[N-1][N-1];
\end{lstlisting}

\begin{description}
    \item[Matrix $\mathbf{L}$:] Accesses on $\mathbf{L}$ have neither spatial nor temporal locality. Since we read the data columnwise.
    \item[Vector $\mathbf{b}$:] Accesses on $\mathbf{b}$ have spatial and temporal locality:
        \begin{itemize}
            \item Accesses on $b[i]$ are temporal.
            \item Accesses on $b[j]$ are spatial.
        \end{itemize}
\end{description}


\end{document}