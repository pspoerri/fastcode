\documentclass[portrait,a4paper]{article}


\usepackage[utf8x]{inputenc}
\usepackage[T1]{fontenc}


\usepackage{mathtools}
\usepackage{amssymb,amsfonts,amsmath}
\usepackage[e]{esvect}

\usepackage{algorithmic}
\usepackage{algorithm}
\newcommand{\algorithmlabel}[2]{{
    \renewcommand{\algorithmicensure}{\textbf{#1}:}
    \ENSURE{#2~}
}}

\usepackage{graphicx}
\usepackage[svgnames]{xcolor}

\usepackage{geometry}
\geometry{a4paper}

% multirow and multicol
\usepackage{multirow}
\usepackage{multicol}
\columnsep24pt
\columnseprule0.1pt

% enumerate
\renewcommand\theenumi{\arabic{enumi}}
\renewcommand\labelenumi{\theenumi.}
\renewcommand\theenumii{\roman{enumii}}
\renewcommand\labelenumii{\theenumii)}

\usepackage{listings}
\lstset{
    floatplacement={tbp},
    basicstyle=\ttfamily\mdseries,
    identifierstyle=,
    stringstyle=\color{gray},
    numbers=left,
    numbersep=5pt,
    inputencoding=utf8x,
    xleftmargin=8pt,
    xrightmargin=8pt,
    keywordstyle=[1]\bfseries,
    keywordstyle=[2]\bfseries,
    keywordstyle=[3]\bfseries,
    keywordstyle=[4]\bfseries,
    numberstyle=\tiny,
    stepnumber=1,
    breaklines=true,
    frame=lines,
    showstringspaces=false,
    tabsize=2,
    commentstyle=\color{gray},
    captionpos=b,
    float=float,
    language={Java}
}
\newcommand{\code}[1]{\lstinline{#1}}

% hyperref
\usepackage[colorlinks=true,pdfborder={0 0 0},citecolor=DarkGreen,linkcolor=DarkBlue,urlcolor=DarkBlue]{hyperref}

% depth of section numbering
\setcounter{secnumdepth}{4}

% redefine the \paragraph command:
\makeatletter
\renewcommand\paragraph{\@startsection{paragraph}{4}{0mm}%
    {-\baselineskip}%
    {0.5\baselineskip}%
    {\normalfont\bfseries}%
}%
\makeatother

% new chapter command
\newcommand{\newchapter}{\clearpage\pagebreak}

% theorems
\usepackage{amsthm}
\newtheorem{lemma}{Lemma}[section]
\newtheorem{theorem}{Theorem}[section]
\newtheorem{corollary}{Corollary}[section]
\newtheorem{definition}{Definition}[section]
\newtheorem{remark}{Remark}[section]
\newtheorem{observation}{Observation}[section]
\newtheorem{assumption}{Assumption}[section]
\newtheorem{proposition}{Proposition}[section]

% autoref names
\newcommand{\specialref}[2]{\hyperref[#1]{#2~\ref*{#1}}}
\def\lstlistingautorefname{Listing}
\def\subsubsectionautorefname{Section}
\def\subsectionautorefname{Section}
\def\figureautorefname{Figure}

% parindent
\parindent0px
\parskip3pt

% redefine greek letters
\renewcommand{\phi}{\varphi}
\renewcommand{\epsilon}{\varepsilon}

% shortcuts in math mode
% \newcommand{\bs}{\boldsymbol}
\newcommand{\mc}{\mathcal}
\newcommand{\ds}{\displaystyle}
\DeclarePairedDelimiter\absimpl{\lvert}{\rvert}
\DeclarePairedDelimiter\normimpl{\lVert}{\rVert}
\newcommand{\abs}[1]{\absimpl*{#1}}
\newcommand{\norm}[1]{\normimpl*{#1}}
\newcommand{\argmax}{\operatorname*{arg\,max}}
\newcommand{\argmin}{\operatorname*{arg\,min}}

% number sets
\newcommand{\R}{\mathbb{R}}
\newcommand{\Z}{\mathbb{Z}}
\newcommand{\N}{\mathbb{N}}
\newcommand{\Q}{\mathbb{Q}}
\newcommand{\C}{\mathbb{C}}
\newcommand{\F}{\mathbb{F}}
\newcommand{\LL}{\mathcal{L}}
\newcommand{\powerset}{\mathcal P}

% probabilities
\newcommand{\Prob}[1]{\operatorname{Pr}\left[#1\right]}
\newcommand{\Ex}[1]{\mathbb{E}\left[#1\right]}

% misc
\newcommand{\bigO}[1]{\mc O\left(#1\right)} % big-o notation

\newcommand{\nop}[1]{} % temporarily remove from output




% todo
\usepackage{framed}
\newenvironment{todo}
{\color{DarkRed} \begin{leftbar}}
{\end{leftbar}}
\newcommand{\inlinetodo}[1]{{\textcolor{DarkRed}{ [\textbf{TODO}: #1]}}}
\newcommand{\mat}[1]{\bs{#1}}
\newcommand{\ma}{\mat{A}}
\newcommand{\mb}{\mat{B}}
\newcommand{\mx}{\mat{X}}
\newcommand{\mv}{\mat{V}}
\newcommand{\muu}{\mat{U}}
\newcommand{\md}{\mat{D}}
\newcommand{\ms}{\mat{S}}
\newcommand{\mz}{\mat{Z}}

\newcommand{\vx}{\mat{x}}
\newcommand{\va}{\mat{a}}
\newcommand{\vb}{\mat{b}}
\newcommand{\vu}{\mat{u}}
\newcommand{\vz}{\mat{z}}

\newcommand{\rd}{\R^D}
\newcommand{\rr}[2]{\R^{#1 \times #2}}

\usepackage{placeins}

\newcommand*{\titleSW}[3]{\begingroup% Story of Writing
\raggedleft
\vspace*{\baselineskip}
{\Huge\textbf{#1}}\\[0.7\baselineskip]
{\large\textbf{#2}}\\[0.5\baselineskip]
{\small #3}\par
\endgroup}

\begin{document}

 \author{Pascal Spörri\\pascal@spoerri.io}
 \title{HOWTO WRITE FAST NUMERICAL CODE\\ EXERCISE 3}
 \date{\today}
\maketitle

\section{Cache Mechanics}
Running the code yields in the following memory acceses:

\begin{figure}[H]
    \centering
    \begin{tabular}{r|cc}
    \textbf{Iteration} & \multicolumn{2}{c}{\textbf{Memory Location}}\\
        $i$ & $x[2*i\%5]$ & $y[2*i\%5]$ \\ \hline
        $0$ & $x[0]$ & $y[0]$\\
        $1$ & $x[2]$ & $y[2]$\\
        $2$ & $x[4]$ & $y[4]$\\
        $3$ & $x[1]$ & $y[1]$\\
        $4$ & $x[3]$ & $y[3]$\\
        $5$ & $x[0]$ & $y[0]$\\
        $6$ & $x[2]$ & $y[2]$\\
        $7$ & $x[4]$ & $y[4]$\\
        $8$ & $x[1]$ & $y[1]$\\
        $9$ & $x[3]$ & $y[3]$\\
    \end{tabular}
    \caption{Memory locations accessed per iteration}
\end{figure}

The cache for each loop iteratio is represented here:

\begin{figure}[H]
%\centering
\hspace{-20mm}
{\small
\begin{tabular}{r|c|c|c|c|c|c|c|c|c|c|c}
\textbf{Cache Line} & \multicolumn{11}{c}{\textbf{Iteration}}\\
16 Byte Blocks & \textbf{Init} & \textbf{0} & \textbf{1} & \textbf{2} & \textbf{3} & \textbf{4} & \textbf{5} & \textbf{6} & \textbf{7} & \textbf{8} & \textbf{9} \\ \hline
0 & - & x[0]x[1]& x[0]x[1] & y[4]y[5] & x[0]x[1] & x[0]x[1] & x[0]x[1] & x[0]x[1] & y[4]y[5] & x[0]x[1] & x[0]x[1]   \\
1 & - & -       & x[2]x[3] & x[2]x[3] & x[2]x[3] & x[2]x[3] & x[2]x[3] & x[2]x[3] & x[2]x[3] & x[2]x[3] & x[2]x[3]  \\
2 & - & y[0]y[1]& y[0]y[1] & x[4]x[5] & y[0]y[1] & y[0]y[1] & y[0]y[1] & y[0]y[1] & x[4]x[5] & y[0]y[1] & y[0]y[1] \\
3 & - & -       & y[2]y[3] & y[2]y[3] & y[2]y[3] & x[2]x[3] & y[2]x[3] & y[2]y[3] & y[2]y[3] & y[2]y[3] & y[2]y[3] \\
\end{tabular}
}
%\centering
\caption{Cache content for each iteration step}
\end{figure}
\subsection{Hit/Miss Sequences}
Based on the cache analysis we are able to determine the miss/hit sequences:
\begin{description}
\item[x:]MMMMHHHMMH
\item[y:]MMMMHHHMMH
\end{description}

\subsection{Miss rate for the individual arrays}
Both arrays have the same miss rate:
\begin{description}
\item[x:] ${6\over 10}$
\item[y:] ${6\over 10}$
\end{description}

\subsection{Operational Intensity}
\begin{description}
\item[Flops:] For each iteration we observe $2$ floating point operations. Which makes $10$ floating point operations for the total run of the program.
\item[Bytes Transferred:] We observe $6$ cache misses per array per execution. This creates a total of $2\cdot 6\cdot 16 = 192$ bytes transferred into the cache.
\end{description}

Thus we can conclude that the total operational intensity is:
\begin{align*}
I = {\text{Flops} \over \text{Bytes Transferred}} = {10\over 192} \approx 0.05
\end{align*}

\section{Cache Mechanics}
For this exercise we evaluate the following code:
\begin{lstlisting}
double x[128], sum;
int i,j;
for (int i=0; i<64; i++) {
    j = i+64;
    sum += x[i]*x[j];
}
\end{lstlisting}

\subsection{Case 1}
\begin{description}
\item[Cache size: $512$ bytes:]
    Since the cache has a size of $512$ bytes and a set size of $16$ bytes we observe that there are $32$ sets.
Both \lstinline{x[i]} and \lstinline{x[j]} will access the set $i\% 2$ and evict themselves on each loop iteration. We can therefore conclude that the miss rate is $100\%$.
\item[Cache size: $1024$ bytes:] \lstinline{x[i]} will access the set $i\% 2$ and \lstinline{x[j]} will access the set $i\% 2 + 32$ on each iteration step. We can therefore conclude that the miss rate is $50\%$.

\end{description}

\subsection{Case 2}
\begin{description}
\item[Cache size: $512$ bytes with $2$ way set associativity] As we have seen before both \lstinline{x[i]} and \lstinline{x[j]} will access the set $i\% 2$. Since we now have $2$ way set associativity the cache lines won't evict each other. We can therefore conclude that the miss rate is $50\%$. 
\item[Increasing the cache] Increasing the cache won't help in this case since the limiting factor is the size of the set: Every two cycles a new set needs to be loaded which will create a miss. 
\item[Increasing the cache line] Increasing the cache line and keeping the set associativity will definitely help since the memory accesses won't compete for a cache line. Doubling the cache line size to $32$ bytes will yield a miss rate of $25\% $.
\end{description}

\section{MMM Analysis}

For this exercise we consider the following code:
\begin{lstlisting}
int i, j, k;
for (i = 0; i < n; i++) {
    for (j + 0; j < n; j++) {
        for (k = 0; k < n; k++) {
            C[i][j] = C[i][j] + A[i][j]*B[k][j];
        }
    }
} 
\end{lstlisting}

\subsection{Smallest Cache Size needed for MMM}
We first proceed to describe the minimal space requirements for each matrix:
\begin{description}
\item[A] Since we need to access each row of matrix $A$ multiple times, we want to hold the row for the outermost row in the cache. The space requirements for the matrix $A$ are therefore: $n\cdot 8$ bytes.
\item[B] We need to access each column of matrix $B$ for each innermost loop iteration. To avoid misses we therefore want to keep the entire matrix $B$ in the cache. The space requirements for matrix $B$ are therefore: $n\cdot n \cdot 8$ bytes.
\item[C] For the matrix $C$ it is sufficient to keep one block in the cache: The accesses are coalesced and the values are reused in each inner loop iteration.
\end{description}
For a fixed $n$ we would therefore need a minimal cache size of $(n+n\cdot n)\cdot 8 + B$ bytes.

\subsection{Largest n for 8 MB cache}
In the $8$ MB cache we can store store $1048576$ doubles (cache line: $8$ doubles).
The largest $n$ that we can use to avoid compulsory misses is $1023$:
\begin{itemize}
    \item We need to use at least $1032$ doubles to store each row of matrix $A$ ($1024$ for the values in the current line $+$ padding since we can't selectively load data and for some cases we need an entire additional block). 
    \item We need to use one block of $8$ doubles to store the the doubles for matrix $C$. 
    \item Matrix $B$ needs $1046529$ doubles to store the entire matrix. With block padding this results in $1046536$ doubles stored for matrix $B$.
\end{itemize}

$1032+ 1046529 + 8= 1047576$ doubles total space needed and thus making a bigger array size impossible (there are only $8000$ bytes to spare).

\subsection{Operational Intensity}
Since every element is only loaded once we can concluded that the total bytes requested by the program are: $3\cdot n^2 \cdot 8= 3\cdot 1024^2\cdot 8 = 25116696$ bytes. Since only the matrix $C$ is written back to memory the program needs to store $8\cdot n^2 =8\cdot 31^2 = 8372232$ bytes. Thus making the total memory traffic: $4\cdot 8372232 = 33488928$ bytes or $31$ MBytes. 

A run of the program yields $2\cdot n^3$ floating point operations. For the concrete example above this results in a total of $1070599167$ floating point operations. 

From the total memory traffic and the floating point operations we are therefore able to compute the operational intensity for the concrete example:
\begin{align*}
    I_{32} = {1070599167\over 8372232} = 127 \left[{\text{Flops} \over \text{Bytes}}\right]
\end{align*}

\section{Proofreading}




\end{document}